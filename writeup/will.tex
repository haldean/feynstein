The Feynstein system is broken into two distinct portions; the
translator and the Feynstein execution environment. The translator
takes, as input, Feynstein source code and translates it into valid
Java source code. This Java code is then run in the Feynstein
execution environment, which pulls in all of the Java code we have
written to provide the rendering, geometry and simulation constructs
that are present in Feynstein.

\subsection{Translator}

The translator uses a multi-pass system to translate Feynstein into
Java. It first does basic lexing, in which it breaks inputs first into
statements, then into the lexemes within those statements. After
lexing, it constructs a syntax tree. Once the code exists within this
tree, the translator walks the tree to generate valid Java code. In
most cases, this is simple textual substitution; for example, the
Feynstein code segment \code{\#myShape} translates directly into
\code{getShape(``myShape'')}. However, some constructs require a more
in-depth parsing; in particular, builder syntax was quite difficult to
translate. These statements, of the form
\code{MyObject(attribute1=value, attribute2=value)}, were not valid
Java and thus didn't cause a collision in our grammar, but still were
difficult to distinguish from, say, \code{MyObject(x == y)}, which is
valid Java (if \code{MyObject} takes a boolean as an argument to its
constructor).

We do have one unusual step in our translator, which is ``structure
analysis'' -- since Feynstein has certain parts of the language which
this user is required to use (for example, the user must specify a
``shapes'' block), we analyze the syntax tree to ensure that they've
incorporated the required portions of the source code.

Error handling in Feynstein is difficult; we need to handle errors
returned by the Java compiler and virtual machine and give users some
indication as to where in their original Feynstein (instead of the
generated Java code) the error occured. To do so, during parsing and
syntax analysis we generate a mapping of Java line numbers to
Feynstein line numbers. Then, if the Java compiler or VM throws an
error, we mine the stack trace for the Java line number which causes
the error and tell the user which line number in Feynstein generated
that Java code.

Both Rob and Will contributed to the translator. Will wrote the
initial implementation of the translator and was responsible for
implementing new features at the translator level. Rob was responsible
for error handling.

A block diagram of the translator is shown in \figref{translator}

\imgfig{translator}{.8}{Block diagram of the stages in the
  translator}{translator}

\subsection{Execution Environment}

The execution environment of Feynstein is the more complex half of the
system; while our language is very similar to Java and therefore
requires simple translations, the actual behavior of a Feynstein
program is extremely complex. This complexity is mirrored in the
complexity of the architecture in the two portions; while the
translator had four modules and a fairly simple API, the execution
environment is composed of almost 7000 lines of Java code split across
50 classes. A block diagram of the execution environment is shown in
\figref{eeblock}.

Will defined the API for the components in the system to ensure
consistency and compatibility with translator output. From a
high-level implementation perspective, Sam was responsible for
rendereing, geometry, forces, and time integration, Colleen was responsible for collision 
detection and response, and Will was responsible for shapes. For a more
in-depth summary of who was responsible for what, see Appendix
\ref{sec:whowhat}, and for a line-by-line listing of who wrote what
code, see Appendix \ref{sec:listing}.

\imgfig{eeblock}{.35}{Block diagram of the execution environment
  architecture}{eeblock}

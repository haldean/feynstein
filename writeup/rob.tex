\subsection{Source Code Management}

In order to manage our team's source code, we used Git, which is a
distributed version control system. Will set up a Git repository on
his server, to which everyone on the team was given SSH access. It
allowed all of us to simultaneously work on Feynstein and then easily
merge our changes together. Git also allowed us to easily track
peoples' contributions to the code base, as it allows a user to see
exactly who wrote each line of code in a repository. Will wrote a
shell script that allowed us to see exactly who wrote how much, as
well, which helped for the final calculations to ensure that everyone
had written at least 500 lines of code.

\subsection{Languages, Tools and Libraries Used}

We wrote our execution environment in Java, as this was the only
language with which the entire team was familiar. It also provided us
with the portability inherent in Java, which was important for a team
of two Mac users, a Windows user and two Linux users.

We make heavy use of OpenGL for our graphics, but there isn't a
native OpenGL implementation in Java, as it requires many
``bare-metal'' computations that aren't possible in a virtual
machine. Instead, we used Java OpenGL\footnote{
 \url{http://jogamp.org}} (or ``JOGL''), which uses JNI to call native
C libraries on the machine. JNI is notorious for being difficult to
set up, and JOGL lived up to this expectation; while we found it was
easy to install for the members of our team who use Linux, it was
extremely difficult to install on OSX. 

However, the installation procedure was worth it; JOGL gave us the
full power of OpenGL, and allowed us to use OpenGL at near-native
speeds and efficiency. Without it, this project would not have been
possible.

On the translator side, we used
PLY\footnote{http://www.dabeaz.com/ply/} (Python Lex-Yacc) for lexing
and parsing. We wanted to use an implementation of Yacc in a scripting
language, as most of the actions in our parser were string
manipulations, and string processing tends to be easier in scripting
languages. We used Python as this is what Will and Rob were more
familiar with.

The Feynstein language evolved rather smoothly throughout the course
of this project. From the very beginning we agreed upon a syntax and
feature set for our language. We defined a set of clear milestones
that we then organized into a language pipeline. Naturally, our first
goal was to have our parser and syntax interpreter up and running, so
our execution code could be written with few changes to the translator
and interpreter. After this, our second milestone was to implement our
renderer, without which no testing could be performed as Feynstein
requires visual display. Both of these hurdles were overcome in the
first half of the semester. With this foundation established, we were
able to divide our language into independent features and assign
development accordingly.

The dependency graph shown in Figure \figref{depend} shows the work
flow for our language development. An arrow indicates a dependency,
i.e. if A->B, then A must be completed before B. Features shown in
italics were deemed second priority because no other language modules
depend upon them and they are not necessary for testing, or there
exists an equivalent feature. Based on this flow graph, we decided to
assign integration and shapes to different team members as they could
clearly be completed in parallel. The team member responsible for
integration would first complete one integrator before developing an
additional integrator, so the person responsible for forces could
begin their work. The person responsible for forces would first
complete spring forces, as collision response will depend on this
logic (and it is the simplest force potential). Forces where An
additional person would then be responsible for handling collisions,
which we agreed upon as a second completion state. As collision
detection extends all other primary modules, it could be left for
future development with the language in a completed state under
undesirable circumstances.

\imgfig{depend}{.5}{Dependencies in Feynstein}{depend}

Our syntax, which is intentionally close to Java for easy integration
of Java code into Feynstein programs, improves upon Java with a
feature we called Builder Syntax. Builder Syntax allows users to
specify Feynstein object instantiation parameters in an arbitrary
order with key-value pairs. For example, a user creates a gravity
force as \code{force GravityForce(gy=-9.8)}. The GravityForce takes
additional optional parameters gx, and gz, which specify the
acceleration due to gravity in the x and z directions,
respectively. This important feature makes Feynstein accessible to
users without requiring an in-depth understanding of its API.

The systems architect took primary responsibility for developing the
translator logic of Builder Syntax translator was completed in the
first half of the project timeline. Meanwhile, the language guru wrote
the OpenGL pipeline render as a stand-alone Java application to be
incorporated into the make Feynstein program. The architect's Builder
Syntax implementation allowed for each development of new-key value
pairs without modifying the basic grammar. Instantiation parameters
are configured as a chain of function calls, each of which returns the
updated object instance. The syntax was also developed to support list
configuration when setting multidimensional parameters; for example,
\code{shape Sphere(position=(0.0,0.2,0.2))}. Builder Syntax keys
translate to methods of name \code{set\_<key\_name>} that return an
instance of the class type--this facilitated rich feature development
for the rest of the team.

With the goal of allowing the team to easily develop properties,
shapes, and forces independently, our architect wrote the APIs for
each base class. With our code checked into a central repository, the
other team members could fill in the method logic
accordingly. Geometric primitives were developed for the renderer and
used to build a triangle mesh class for shape topology. As planned, we
developed our initial custom shapes in parallel with the Semi-implict
Euler integrator. We implemented the .obj file parser first, as this
would allow for quick render testing without defining complex
geometric configurations dynamically. We then developed forces while
extending our custom shape library. Forces were developed in order of
increasing complexity measured in the number of particles they act
upon. The most complex force we aspired to support, a tetrahedral
constraint force, was omitted from the language as it only applied to
a single shape configuration. We developed particle-Interpolation
shapes, which are small shapes defined by a list of vertices, in light
of forces for easy unit testing of force stencils.

We had hoped to develop a second time integration class in parallel to
force and shape development, but did not reach this goal. The
ImplicitEuler method we hoped to support required a complex
multi-dimensional numerical solver for which we could not find an
appropriate external library. We did however, add a less complex
Velocity-Verlet based integrator to offer users more flexibility. We
completed all force and shape definitions soon enough that we could
focus our last efforts towards building our collision detection system
while simultaneously developing a syntactical feature that allowed
users to manipulate shapes with predefined operators. We first
completed proximity-based collision detection, then continuous-time
collision detection, which is significantly more complicated and
precise. We then developed a penalty-spring-based collision handler
followed by an impulse-based handler.

With the exception of our two minor changes, our language evolved
beyond our first completion stage. We achieve our milestones on time,
though sacrifices in division of labor often had to be made to reach
this goal. With better deadline achievement, I believe we could have
achieved more interesting rendering capabilities into the
languages--such as lighting and textures. Nevertheless, our final
feature set provides a rich framework for configuring physics-based
simulation as well as many easily-extendable building blocks for
future research and development.
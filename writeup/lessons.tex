\subsection{Team Lessons}
We discovered the importance for having a central repository for all
updates to our grammar, of which there were many and which were
communicated between members of the group instead of being recorded.

We learned the importance of defining general responsibilities in
advance: our roles were far less discrete than we expected them to be,
and while this is indicative of the cooperative way we approached this
project, it led to an unequal division of work that could have
possibly been avoided.  

We learned that the more modular our language was, the easier it was
to add--and remove--language components easily. 

\subsection{Individual Lessons}
\begin{description}
\item[Colleen: Project Manager] I initially approached the project as
  a very lenient manager, assuming that my group members--myself
  included--would produce work perfectly and according to schedule. I
  learned quickly that expecting people to behave as predictably as
  machines was unrealistic, and I began to understand the importance
  of instating policies for less-than-ideal situations at the
  beginning of the project, rather than once those situations have
  already happened. I also learned the importance of understanding
  different styles of communication and of figuring out how one’s team
  members work best. By the end of our project, in fact, I had
  discovered that I got the best results from my team by contacting
  one team member far less frequently than I’d have liked to, but also
  contacting another member by phone almost daily--both of which were
  unintuitive ways for me to communicate. My belief in the validity of
  Hofstadter’s Law--”It always takes longer than you expect, even when
  you take into account Hofstadter's Law”--has been confirmed many
  times over in the course of this project. I’ve learned a lot about
  the importance of setting realistic deadlines, but I’ve realized
  that it’s equally important to accept that sometimes a schedule you
  thought was realistic at first needs to be changed.
\end{description}

\subsection{Advice for Future Teams}
I found the existence of our team wiki to be very helpful, and I would
encourage other groups to implement their own and use it even more
extensively than our group did. I would suggest that other groups make
sure that the basic framework for working together on a coding project
is in place early in the project. This framework would include
expectations about meetings and communication in addition to tools to
store, update, and communicate about the compiler’s code at all stages
of the project.

\subsection{Instructor Feedback}
I found the sections on run-time environments (chapter 7) code
generation (chapter 8) to be particularly interesting--both gave me a
far greater understand of facets of computer programming that I had
previously taken for granted.

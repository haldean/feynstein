\subsection{Team Lessons}
We discovered the importance for having a central repository for all
updates to our grammar, of which there were many and which were often
communicated between members of the group instead of being recorded.

We learned the importance of defining general responsibilities in
advance: our roles were far less discrete than we expected them to be,
and while this is indicative of the cooperative way we approached this
project, it led to an unequal division of work that could have
possibly been avoided.  

We learned that the more modular our language was, the easier it was
to add--and remove--language components easily. 

\subsection{Individual Lessons}
\begin{description}
\item[Colleen: Project Manager] My belief in the validity of
  Hofstadter's Law--"It always takes longer than you expect, even when
  you take into account Hofstadter's Law"--has been confirmed many
  times over in the course of this project. I've learned a lot about
  the importance of setting realistic deadlines, but I've realized
  that it's equally important to accept that sometimes a schedule you
  thought was realistic at first needs to be radically changed.

  On a more individual note: I also learned the importance of understanding
  different styles of communication and of figuring out how one's team
  members work best: I believe that a few pitfalls could have
  been avoided if I had worked on our group's communication practices
  earlier in the project timeline. I initially approached the project as
  a very lenient manager, assuming that my group members--myself
  included--would produce work perfectly and according to schedule. I
  learned quickly that expecting people to behave as predictably as
  machines was unrealistic, and I began to understand the importance
  of instating policies for unfavorable situations at the
  beginning of the project, rather than once those situations have
  already happened.

  For our group, these situations mostly involved missed deadlines and an
  ensuing redistribution of work to the group members most able and willing
  to complete it. I began the project with the goal of giving all
  group members considerable enough contributions that they would
  all feel a sense of ownership for the project, but ultimately this
  was now how our group functioned, and I learned that at a certain
  point it's necessary to assign work to whoever will get it done
  quickly. A closely related lesson--and the hardest one for me to
  accept--was that sometimes the Project Manager has to prioritize
  completing a project over being nice or trying to keep everyone happy.

\item[Sam: Language Guru] Working on this language was fascinating for
  me, especially in a group dynamic. I proposed this language idea to
  my teammates because physical simulation is my primary area of
  research. The idea was very much inspired by my own experience
  learning simulation. I took a course in physical simulation that was
  taught entirely in C++, a language I was unfamiliar with at the
  time. I found (and I was far from alone) that most of my energy was
  put towards learning to use a new language in a very complex way:
  building an interactive object-oriented system and optimizing for
  performance. Although this was a great learning experience, within
  that particular course, I wished I could have focused more on
  understanding the physics behind the engine I was writing rather
  than just getting it to work--so much time was spent on the initial
  implementation that little was left for
  experimentation. As a result, I was inspired to create a language
  that facilitated easy understanding of the physical simulation
  concepts, but also allowed for more complex extension and
  experimentation.

  Needless to say, there was great pedagogical merit in this
  process. I am interesting in learning how different people
  understand the task of programming physical systems, and this gave
  me great insight. The language design is very much a reflection of
  this pedagogical experience. The act of teaching new concepts
  involves abstracting the lower level details into comprehensible
  objects and relationships. That said, my initial lessons for
  teaching simulation to the team ended up defining our language
  model. Furthermore, by working in Java, I learned a great deal about
  adapting my own perspective of developing simulation code, which is
  characterized by C++, to new languages--especially in terms of
  achieving optimal performance.

  Simulation aside, another important lesson I learned is about the
  dynamics of team work. In past experience, I have always tried to
  divide up labor as efficiently as possible so the maximal number of
  components could be developed in parallel. This division of labor
  was always straightforward when each person considered themselves a
  unique specialist. But what if not everyone is a specialist? I our
  group, there were some members who came in knowing exactly what they
  wanted to do, and others who were more flexible. There is, of
  course, an appeal to the person who is willing to work on anything,
  but from what I have learned, if someone considers themselves an
  expert, they will perform accordingly. I think that passion makes a
  team, and team members need to really to be confident in their
  necessity within the larger group from the beginning. I've learned
  that when selling an individual vision to a group of teammates, you
  should be certain each person is excited about their role.
	
  On that same token, one of the hardest parts of the language guru's
  job is letting go of control. When you envision a perfect system,
  you can't imagine accepting outside input. In some cases, my
  teammates showed me their clever and new perspectives on methods
  with which I've become rather familiar with developing
  simulators. It was admittedly hard to assign major coding components
  with which I was so familiar to my teammates. Some team members were
  very responsive, and I enjoyed collecting relevant reading materials
  to help them understand larger concepts, helping them through the
  coding process, and seeing their genuine excitement when things were
  working. Unfortunately, in other cases I believe some tasks may have
  been too daunting for team members. In these cases, I learned that
  you have to be aggressive when offering help--people won't always
  ask for it. On the other side, I also learned that sometimes you
  need to be resolute and put the language first, even if that means
  redistributing the work load towards those who consistently complete
  their work on time.

\item[Will: System Architect] I learned the value of defining APIs
  early on, because just the definition of an API is enough to get
  everyone thinking about how exactly things work. We had some issues
  later in the semester in which things couldn't work as we had
  planned, and we would have realized that that was the case if we had
  defined just what methods and fields each class in our execution
  environment had.

  I also learned (from experience) why everyone uses parser generators
  like YACC; at first, I tried to implement our language in pure
  Python, starting from scratch. I found that this approach made it
  much easier to write the actions in the SDTS for our language, but
  it made it extremely difficult to adapt to changes in our language's
  grammar.

  Finally, I learned a lot about my own division of labor when it
  comes to working in a team. Specifically, I found that the most
  challenging part of the project was responding to emails and
  maintaining the administrative part of the project; the coding came
  comparatively easily, despite the fact that I did a lot of
  coding. In fact, the largest headaches came from my job managing the
  documentation -- compiling everyone's contributions, ensuring that
  things got done on time and responding to people's requests for
  edits in a timely fashion were probably the most difficult tasks
  that I had.

\item[Rob: Systems] During this project, I learned two things. The
  first and probably most important thing I learned is that
  communication is essential in successful group work. I learned
  that I do not do this well and that it is something I need to
  improve upon if I want to be a more successful and productive group
  member in the future. The second thing I learned is that starting to
  code early is also really important in getting the best results as
  possible. It allows time to revise the code and make sure it works
  properly. That is also something which I will need to work on.

\item[Eva: Testing] You never know how much you're going to miss a
  debugger until it isn't there anymore.  It's pretty amazing the
  amount of knowledge contained in the heads of 5 Computer Science
  majors, but it's only helpful if you're asking questions.  Teammates
  and classmates can be a great resource, but only when you take the
  initiative to communicate and ask questions.
\end{description}

\subsection{Advice for Future Teams}
We found the existence of our team wiki to be very helpful, and would
encourage other groups to implement their own and use it even more
extensively than our group did. We would suggest that other groups
make sure that the basic framework for working together on a coding
project is in place early in the project. This framework would include
expectations about meetings and communication in addition to tools to
store, update, and communicate about the compiler's code at all stages
of the project. There are few replacements for effective
communication; in some cases, team members missed deadlines just
because their teammates didn't respond to emails in a timely
manner. Starting early and communicating effectively are both key to
meeting deadlines with high-quality results.

We also would advise future teams to set as many bail out points as
possible. Define versions 1.0, 1.1, and 1.2 of your language. Aspire
for version 1.2, but be willing to accept 1.0. This was definitely the
saving grace of our design process: we had many small
milestones. Also, it is important to never hesitate to ask team
members for help. This was definitely a struggling point for our
team. Remember that asking for help isn't a reflection of a lack of
intelligence, and more communication between teammates is always a
good thing. Getting stuck and not telling anyone is never an excuse
for failing to meet a deadline. On that same note: meet and meet
often. Lastly, be certain from the very beginning that every team
member knows the answer to these three questions: what is their role,
why were they chosen for it in particular, and are they excited about
it?

\subsection{Instructor Feedback}
\begin{description}
\item[Sam] I found the material for this course to be as interesting
  as it was difficult to grasp. My first suggestion would be more
  consistent written homework--however small--to reinforce ideas as
  they are learned. I would also suggest more consistency between
  homework and exams. I found that whereas the homework were more
  procedural, deriving from examples in the book, the exams were more
  conceptual.

  As far as the project is concerned, I would suggest a clearer
  definition of the five roles from the beginning. I remember our team
  did not gain a clear understanding of the distinction between the
  systems architect and integrator until the final report descriptions
  were posted. On that note, I often found that project-related
  assignments, such as the white paper and the language manual, did
  not come with explicit description. Perhaps more lessons focusing on
  the language design process would be beneficial.

\item[Colleen] I found the sections on run-time environments (chapter
  7) code generation (chapter 8) to be particularly interesting--both
  gave me a far greater understand of facets of computer programming
  that I had previously taken for granted. 

  As my group's project manager,
  I also loved Bob Martin's presentation on software projects, but I would
  have appreciated the information even more early in the semester, before
  I had already run into some of the problems he warned us about. In general,
  I would have liked to see  more information about project group roles and 
  advice about potential pitfalls early in the semester.

\item[Will] I really enjoyed all of the material associated with this
  course, and I think the project really gave me an appreciation for
  compilers and compiling tools. If I were to change one thing about
  this course, it would be to make homeworks more frequent. I would
  have liked to have more opportunities to apply some of the knowledge
  that I learned in lecture. Also, I'm not sure if this is unique to
  our language or all teams' languages, but we never got into any code
  optimization in Feynstein. I would have found a homework that
  required me to implement some basic code optimization really
  challenging and enjoyable.

\item[Eva] I think we covered grammars and regular languages a little
  more extensively than we needed to given that they were review. It
  would have been nice to get to YACC and lex right away.

\item[Rob] For future classes, I'd like to see a bit more examples of
  other programming languages and syntax styles, for no other reason
  other than I think things like that and I think it would be
  interesting and entertaining.
\end{description}

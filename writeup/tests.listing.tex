\subsection*{tests/compile.f}
\begin{lstlisting}
MyScene {
    shapes {
	shape Cylinder(name="cyl1", radius=10cm, 
		       location=(20mi, 30yd, 0), height=50m);

	// Nested blocks
	if (4 newton + 2.7e10 forcelb == 6 dyne) {
	    print("Something happened");
	}
    }

    /**
     * Creates some super-special forces.
     */
    forces {
	force SpringForce(actsOn=#cyl1, length=10mi, strength=4);
    }

    properties { }
}
\end{lstlisting}

\subsection*{tests/ops.f}
\begin{lstlisting}
OpsScene {
    boolean growing = false;
    int growingCount = 0;

    shapes {
	shape Cube(name="cube", allSides=20);
    }

    forces none;
    properties none;

    onFrame {
	#cube.rotate(0.07, 0.02, 0.11);
	#cube.translate(.1, 0, 0);

	if (growing) {
	    #cube.scale(1.02, 1, 1);
	} else {
	    #cube.scale(.98, 1, 1);
	}

	growingCount++;
	if (growingCount > 100) {
	    growingCount = 0;
	    growing = ! growing;
	}
    }
}\end{lstlisting}

\subsection*{tests/springforce.f}
\begin{lstlisting}
MyScene {
    shapes {
	shape SpringChain(name="obj1", vert=(0,20,0), vert=(0,10,0), vert=(0,0,0), fixed=0);
    }

    /**
     * Creates some super-special forces.
     */
    forces {
	force SpringForce(actsOn=#obj1, strength=50, length=10);
	force GravityForce(gy=-9.8);
    }

    properties { 
	property SemiImplicitEuler(stepSize=0.01);
   }
}
\end{lstlisting}

\subsection*{tests/properties.f}
\begin{lstlisting}
PropertyScene {
  properties {
    property SemiImplicitEuler(stepSize=10ms);
  }

  shapes none;
  forces none;
}
\end{lstlisting}

\subsection*{tests/gravity.f}
\begin{lstlisting}
MyScene {
    shapes {
	shape SinglePointMass(name="obj1", pos=(-2,0,0), mass=1);
	shape SinglePointMass(name="obj1", pos=(0,0,0), mass=2);
	shape SinglePointMass(name="obj1", pos=(2,0,0), mass=3);
    }

    /**
     * Creates some super-special forces.
     */
    forces {
	force GravityForce(gy=-1.0);
    }

    properties { 
	property SemiImplicitEuler(stepSize=0.01);
   }
}
\end{lstlisting}

\subsection*{tests/triangle.f}
\begin{lstlisting}
MyScene {
    shapes {
	shape TriangleShape(name="obj1", vert=(-1,-2,0), vert=(0,-2,0), vert=(-0.5,-1.0,0), fixed=2);
	shape TriangleShape(name="obj2", vert=(1,-2,0), vert=(2,-2,0), vert=(1.5, -1, 0), fixed=2);
	shape TriangleShape(name="obj3", vert=(3,-2,0), vert=(4,-2.0, 0), vert=(3.5,-1,0), fixed=2);
	shape TriangleShape(name="obj4", vert=(5,-2,0), vert=(6,-2,0), vert=(5.5, -1, 0), fixed=2);
        shape TriangleShape(name="obj5", vert=(7,-2,0), vert=(8,-2.0, 0), vert=(7.5,-1,0), fixed=2);

	}

    /**
     * Creates some super-special forces.
     */
    forces {
	force TriangleForce(actsOn=#obj1, youngs=1.0, poisson=0.1);
	force TriangleForce(actsOn=#obj2, youngs=1.0, poisson=0.2);
	force TriangleForce(actsOn=#obj3, youngs=1.0, poisson=0.3);
	force TriangleForce(actsOn=#obj4, youngs=1.0, poisson=0.4);
        force TriangleForce(actsOn=#obj5, youngs=1.0, poisson=0.49);
	force GravityForce(gy=-9.8);
    }

    properties { 
	property SemiImplicitEuler(stepSize=0.001);
   }
}
\end{lstlisting}

\subsection*{tests/rodbending1\_vv.f}
\begin{lstlisting}
rodbending1 {

    shapes {

	shape SpringChain(name="spring1", vert=(0,0,0), vert=(1,0,0),
		vert=(2,0,0), vert=(3,0,0), vert=(4,0,0), vert=(5,0,0),
		vert=(6,0,0), vert=(7,0,0), fixed=0, fixed=1);
    }


    forces {
	
	force RodBendingForce(actsOn=#spring1, strength=1, angle=0);

	force SpringForce(actsOn=#spring1, strength=1000, length=1);

	force GravityForce(gy=-2.5);
	
    }

    properties { 
	property VelocityVerlet(stepSize=0.001);
   }
}

\end{lstlisting}

\subsection*{tests/velocityverlet.f}
\begin{lstlisting}
MyScene {
    shapes {
	shape SpringChain(name="obj1", vert=(0,20,0), vert=(0,10,0), vert=(0,0,0), fixed=0);
    }

    /**
     * Creates some super-special forces.
     */
    forces {
	force SpringForce(actsOn=#obj1, strength=50, length=10);
	force GravityForce(gy=-9.8);
    }

    properties { 
	property VelocityVerlet(stepSize=0.01);
   }
}
\end{lstlisting}

\subsection*{tests/rodbending1.f}
\begin{lstlisting}
rodbending1 {

    shapes {

	shape SpringChain(name="spring1", vert=(0,0,0), vert=(1,0,0),
		vert=(2,0,0), vert=(3,0,0), vert=(4,0,0), vert=(5,0,0),
		vert=(6,0,0), vert=(7,0,0), fixed=0, fixed=1);
    }


    forces {
	
	force RodBendingForce(actsOn=#spring1, strength=1, angle=0);

	force SpringForce(actsOn=#spring1, strength=1000, length=1);

	force GravityForce(gy=-2.5);
	
    }

    properties { 
	property SemiImplicitEuler(stepSize=0.001);
   }
}

\end{lstlisting}

\subsection*{tests/cantilever.f}
\begin{lstlisting}
MyScene {
    shapes {
	//shape CustomObject(name="obj1", file="scenes/BunnyCrusher.obj");
	shape SpringChain(name="obj1", vert=(0,1,0), vert=(0,0,0), vert=(1,1,0), vert=(1,0,0), 
	vert=(2,1,0), vert=(2,0,0), vert=(3,1,0), vert=(3,0,0), vert=(4,1,0), vert=(4,0,0), connectivity=3);
	// Nested blocks
	if (4 newton + 2.7e10 forcelb == 6 dyne) {
	    print("Something happened");
	}
    }

    /**
     * Creates some super-special forces.
     */
    forces {
	force SpringForce(actsOn=#obj1, strength=10);
	//force TriangleForce(actsOn=#obj1, youngs=1, poisson=0.1);
	//force SurfaceBendingForce(actsOn=#obj1);
	//force GravityForce(gy=-9.8);
    }

    properties { 
	property SemiImplicitEuler(stepSize=0.01);
   }
}
\end{lstlisting}

\subsection*{tests/rodbending0.f}
\begin{lstlisting}
rodbending0 {

    shapes {

	shape SpringChain(name="spring1", vert=(0,0,0), vert=(0,1,0),
		vert=(0,2,0), mass=1);

	shape SpringChain(name="spring2", vert=(2,0,0), vert=(2,1,0),
		vert=(2.707106781186548,1.707106781186548,0), mass=1);

	shape SpringChain(name="spring3", vert=(-2,0,0), vert=(-2,1,0),
		vert=(-3,1,0), mass=1);

	shape SpringChain(name="spring4", vert=(-2,-3,0), vert=(-2,-2,0),
		vert=(-3,-2,0), mass=1);
	shape SpringChain(name="spring5", vert=(0,-3,0), vert=(0,-2,0),
		vert=(-1,-2,0), mass=2);
	shape SpringChain(name="spring6", vert=(2,-3,0), vert=(2,-2,0),
		vert=(1,-2,0), mass=4);

	shape SpringChain(name="spring7", vert=(-2,-5,0), vert=(-2,-4,0),
		vert=(-3,-4,0), mass=1);
	shape SpringChain(name="spring8", vert=(0,-5,0), vert=(0,-4,0),
		vert=(-1,-4,0), mass=1);
	shape SpringChain(name="spring9", vert=(2,-5,0), vert=(2,-4,0),
		vert=(1,-4,0), mass=1);

	shape SpringChain(name="spring10", vert=(-3,-7,0), vert=(-2,-7,0),
		vert=(-2,-8,0), mass=1);
	shape SpringChain(name="spring11", vert=(-1,-7,0), vert=(0,-7,0),
		vert=(-0,-8,0), mass=1);
	shape SpringChain(name="spring12", vert=(1,-7,0), vert=(2,-7,0),
		vert=(2,-8,0), mass=1);

	
    }


    forces {
	
	force RodBendingForce(actsOn=#spring1, strength=1, angle=0);
	force RodBendingForce(actsOn=#spring2, strength=1,
		angle=0.785398163397448);
	force RodBendingForce(actsOn=#spring3, strength=1,
		angle=1.570796326794897);

	force RodBendingForce(actsOn=#spring4, strength=1, angle=0);
	force RodBendingForce(actsOn=#spring5, strength=2, angle=0);
	force RodBendingForce(actsOn=#spring6, strength=4, angle=0);

	force RodBendingForce(actsOn=#spring7, strength=1, angle=0);
	force RodBendingForce(actsOn=#spring8, strength=2, angle=0);
	force RodBendingForce(actsOn=#spring9, strength=4, angle=0);

	force RodBendingForce(actsOn=#spring1, strength=1,
		angle=0.3926990816987242);
	force RodBendingForce(actsOn=#spring1, strength=1,
		angle=0.4487989505128276);
	force RodBendingForce(actsOn=#spring1, strength=1,
		angle=0.5235987755982989);


	force SpringForce(actsOn=#spring1, strength=5000, length=1);
	force SpringForce(actsOn=#spring2, strength=5000, length=1);
	force SpringForce(actsOn=#spring3, strength=5000, length=1);

	force SpringForce(actsOn=#spring4, strength=5000, length=1);
	force SpringForce(actsOn=#spring5, strength=5000, length=1);
	force SpringForce(actsOn=#spring6, strength=5000, length=1);

	force SpringForce(actsOn=#spring7, strength=5000, length=1);
	force SpringForce(actsOn=#spring8, strength=5000, length=1);
	force SpringForce(actsOn=#spring9, strength=5000, length=1);


    }

    properties { 
	property SemiImplicitEuler(stepSize=0.001);
   }
}
\end{lstlisting}

\subsection*{tests/particlevelocity.f}
\begin{lstlisting}
MyScene {
    shapes {
	shape TriangleShape(name="obj1", vert=(-1,-2,0), vert=(0,-2,0), vert=(-0.5,-1.0,0));
	shape TriangleShape(name="obj2", vert=(1,-2,0), vert=(2,-2,0), vert=(1.5, -1, 0), 
	velocity=(0,0,0.1,0), velocity=(1,0,0.1,0), velocity=(2,0,-0.2,0));
	shape TriangleShape(name="obj3", vert=(3,-2,0), vert=(4,-2.0, 0), vert=(3.5,-1,0),
	velocity=(0,0,0.1,0), velocity=(1,0,0.1,0), velocity=(2,0,-0.2,0));
	shape TriangleShape(name="obj4", vert=(5,-2,0), vert=(6,-2,0), vert=(5.5, -1, 0),
	velocity=(0,0,0.1,0), velocity=(1,0,0.1,0), velocity=(2,0,-0.2,0));
        shape TriangleShape(name="obj5", vert=(7,-2,0), vert=(8,-2.0, 0), vert=(7.5,-1,0),
	velocity=(0,0,0.1,0), velocity=(1,0,0.1,0), velocity=(2,0,-0.2,0));
	}

    /**
     * Creates some super-special forces.
     */
    forces {
	force TriangleForce(actsOn=#obj1, youngs=1.0, poisson=0.1);
	force TriangleForce(actsOn=#obj2, youngs=1.0, poisson=0.2);
	force TriangleForce(actsOn=#obj3, youngs=1.0, poisson=0.3);
	force TriangleForce(actsOn=#obj4, youngs=1.0, poisson=0.4);
        force TriangleForce(actsOn=#obj5, youngs=1.0, poisson=0.49);
	//force GravityForce(gy=-9.8);
    }

    properties { 
	property SemiImplicitEuler(stepSize=0.001);
   }
}
\end{lstlisting}

\subsection*{tests/initvel.f}
\begin{lstlisting}
InitVelocity {
    shapes {
	shape Sphere(name="s", radius=20, velocity=(0,20,0));
    }

    forces {
	force GravityForce(gy=-10);
    }

    properties {
	property SemiImplicitEuler(stepSize=0.01);
    }
}\end{lstlisting}

\subsection*{tests/collide.f}
\begin{lstlisting}
CollideScene {
    shapes {
	shape FluidPlane(name="plane", length=60, subdivisions=10,
			 location=(-10, -10, 60), mass=10kg);
	shape Sphere(name="sphere", radius=20, accuracy=10, fixed=true);
    }

    forces {
	force GravityForce(gz=-9.8);
    }

    properties {
	property SemiImplicitEuler(stepSize=0.01);
	property BoundingVolumeHierarchy(type=BoundingVolumeHierarchy.AABB, margin=0.1);
    }
}
\end{lstlisting}

\subsection*{tests/surfacebendingforce.f}
\begin{lstlisting}
MyScene {
    shapes {
	//shape CustomObject(name="obj1", file="scenes/BunnyCrusher.obj");
	shape ClothPiece(name="obj1", vert=(0,-1,0), vert=(-1,-2,-1), vert=(0,-3,0), vert=(1,-2,-1), 
	vert=(3,-1,0), vert=(2,-2,-1), vert=(3,-3,0), vert=(4,-2,-1));
	// Nested blocks
	if (4 newton + 2.7e10 forcelb == 6 dyne) {
	    print("Something happened");
	}
    }

    /**
     * Creates some super-special forces.
     */
    forces {
	//force TriangleForce(actsOn=#obj1, youngs=500, poisson=0.1);
	force SurfaceBendingForce(actsOn=#obj1, strength=.1);
	//force GravityForce(gz=0.05);
    }

    properties { 
	property SemiImplicitEuler(stepSize=0.01);
   }
}
\end{lstlisting}

\subsection*{tests/springforce1.f}
\begin{lstlisting}
springforce1 {
    shapes {
	int m = 1;
	shape SpringChain(name="spring1", vert=(0,0,0), vert=(0,1,0), mass=m);
	shape SpringChain(name="spring2", vert=(1,0,0), vert=(1,2,0), mass=m);
	shape SpringChain(name="spring3", vert=(2,0,0), vert=(2,3,0), mass=m);
	shape SpringChain(name="spring4", vert=(3,0,0), vert=(3,4,0), mass=m);
	shape SpringChain(name="spring5", vert=(4,0,0), vert=(4,5,0), mass=m);

	shape SpringChain(name="spring6", vert=(0,-1,0), vert=(0,-2,0), mass=1);
	shape SpringChain(name="spring7", vert=(1,-1,0), vert=(1,-2,0), mass=2);
	shape SpringChain(name="spring8", vert=(2,-1,0), vert=(2,-2,0), mass=3);
	shape SpringChain(name="spring9", vert=(3,-1,0), vert=(3,-2,0), mass=4);
	shape SpringChain(name="spring10", vert=(4,-1,0), vert=(4,-2,0),
		mass=5);

	shape SpringChain(name="spring11", vert=(0,-3,0), vert=(0,-4,0),
		mass=m);
	shape SpringChain(name="spring12", vert=(1,-3,0), vert=(1,-4,0),
		mass=m);
	shape SpringChain(name="spring13", vert=(2,-3,0), vert=(2,-4,0),
		mass=m);
	shape SpringChain(name="spring14", vert=(3,-3,0), vert=(3,-4,0),
		mass=m);
	shape SpringChain(name="spring15", vert=(4,-3,0), vert=(4,-4,0),
		mass=m);

	shape SpringChain(name="spring16", vert=(0,-5,0), vert=(0,-6,0),
		mass=m);
	shape SpringChain(name="spring17", vert=(1,-5,0), vert=(1,-6,0),
		mass=m);
	shape SpringChain(name="spring18", vert=(2,-5,0), vert=(2,-6,0),
		mass=m);
	shape SpringChain(name="spring19", vert=(3,-5,0), vert=(3,-6,0),
		mass=m);
	shape SpringChain(name="spring20", vert=(4,-5,0), vert=(4,-6,0),
		mass=m);
    }

    forces {
	force SpringForce(actsOn=#spring1, strength=10, length=1);
	force SpringForce(actsOn=#spring2, strength=10, length=2);
	force SpringForce(actsOn=#spring3, strength=10, length=3);
	force SpringForce(actsOn=#spring4, strength=10, length=4);
	force SpringForce(actsOn=#spring5, strength=10, length=5);

	force SpringForce(actsOn=#spring6, strength=1, length=0.8);
	force SpringForce(actsOn=#spring6, strength=2, length=0.8);
	force SpringForce(actsOn=#spring6, strength=3, length=0.8);
	force SpringForce(actsOn=#spring6, strength=4, length=0.8);
	force SpringForce(actsOn=#spring6, strength=5, length=0.8);

	force SpringForce(actsOn=#spring6, strength=1, length=0.8);
	force SpringForce(actsOn=#spring7, strength=2, length=0.8);
	force SpringForce(actsOn=#spring8, strength=4, length=0.8);
	force SpringForce(actsOn=#spring9, strength=8, length=0.8);
	force SpringForce(actsOn=#spring10, strength=16, length=0.8);

	force SpringForce(actsOn=#spring11, strength=1, length=0.9);
	force SpringForce(actsOn=#spring12, strength=1, length=0.8);
	force SpringForce(actsOn=#spring13, strength=1, length=0.7);
	force SpringForce(actsOn=#spring14, strength=1, length=0.6);
	force SpringForce(actsOn=#spring15, strength=1, length=0.5);

    }

    properties { 
	property SemiImplicitEuler(stepSize=0.01);
   }
}
\end{lstlisting}

\subsection*{tests/shapetests.f}
\begin{lstlisting}
ShapeScene {
  shapes {
      shape Cube(name="cube", sides=(20,30,40), location=(40,0,0));
      shape Cylinder(name="cyl", radius=10, height=40, location=(-40,0,0));
      shape Cylinder(name="cyl2", radius1=5, radius2=20, height=50, location=(0,40,0));
      shape Tetrahedron(name="tetra", point1=(0,0,0), point2=(20,0,0), 
      			point3=(0,20,0), point4=(0,0,20), location=(0,0,-40));
      shape RegularPolygon(name="penta", radius=10, verteces=5, location=(0,-40,0));
      shape Sphere(name="sph", radius=10);
      shape FluidPlane(name="cloth", lengthX=20, lengthY=20, subdivisions=40,
		       location=(-40,-40,0));
  }
  
  forces none;
  properties none;
}
\end{lstlisting}

\subsection*{tests/transtest.f}
\begin{lstlisting}
springforce1 {
    shapes {
	//java code works?
	if(1==0)  {
		System.out.println("error");
	}
	/**
	 * all types of comments work?
	 */

	shape SpringChain(name="spring1", vert=(0,-3,0), vert=(0,-4,0),
		mass=1);
	shape SpringChain(name="spring2", vert=(1,-3,0), vert=(1,-4,0),
		mass=1);
	shape SpringChain(name="spring3", vert=(2,-3,0), vert=(2,-4,0),
		mass=1);
	shape SpringChain(name="spring4", vert=(3,-3,0), vert=(3,-4,0),
		mass=1);
	shape SpringChain(name="spring5", vert=(4,-3,0), vert=(4,-4,0),
		mass=1);
    }
	if(0==1)  {
		System.out.println("error");
	}

    forces {

	force SpringForce(actsOn=#spring1, strength=10, length=1);
	force SpringForce(actsOn=#spring2, strength=10, length=2);
	force SpringForce(actsOn=#spring3, strength=10, length=3);
	force SpringForce(actsOn=#spring4, strength=10, length=4);
	force SpringForce(actsOn=#spring5, strength=10, length=5);

    }

    properties { 

	property SemiImplicitEuler(stepSize=0.01);
   }

    onFrame {

	System.out.println("Rendered frame.");

   }   

}
\end{lstlisting}

\subsection*{tests/springforce0.f}
\begin{lstlisting}
springforce0 {

    shapes {
	int m = 1;
	//Fix one end, increase k.  Weaker springs should stretch more.
	shape SpringChain(name="spring1", vert=(0,0,0), vert=(0,1,0),
		mass=m, fixed=1);
	shape SpringChain(name="spring2", vert=(1,0,0), vert=(1,1,0),
		mass=m, fixed=1);
	shape SpringChain(name="spring3", vert=(2,0,0), vert=(2,1,0),
		mass=m, fixed=1);
	shape SpringChain(name="spring4", vert=(3,0,0), vert=(3,1,0),
		mass=m, fixed=1);
	shape SpringChain(name="spring5", vert=(4,0,0), vert=(4,1,0),
			mass=m, fixed=1);

	//Fix one end, increase m.  More mass should stretch more.
	shape SpringChain(name="spring6", vert=(0,-7,0), vert=(0,-8,0),
		mass=1, fixed=0);
	shape SpringChain(name="spring7", vert=(1,-7,0), vert=(1,-8,0),
		mass=2, fixed=0);
	shape SpringChain(name="spring8", vert=(2,-7,0), vert=(2,-8,0),
		mass=3, fixed=0);
	shape SpringChain(name="spring9", vert=(3,-7,0), vert=(3,-8,0),
		mass=4, fixed=0);
	shape SpringChain(name="spring10", vert=(4,-7,0), vert=(4,-8,0),
		mass=5, fixed=0);
    }


    forces {
	force SpringForce(actsOn=#spring1, strength=1, length=1);
	force SpringForce(actsOn=#spring2, strength=2, length=1);
	force SpringForce(actsOn=#spring3, strength=4, length=1);
	force SpringForce(actsOn=#spring4, strength=8, length=1);
	force SpringForce(actsOn=#spring5, strength=16, length=1);

	force SpringForce(actsOn=#spring6, strength=1, length=1);
	force SpringForce(actsOn=#spring7, strength=1, length=1);
	force SpringForce(actsOn=#spring8, strength=1, length=1);
	force SpringForce(actsOn=#spring9, strength=1, length=1);
	force SpringForce(actsOn=#spring10, strength=1, length=1);
    }

    properties { 
	property SemiImplicitEuler(stepSize=0.01);
   }
}
\end{lstlisting}

\subsection*{tests/test.f}
\begin{lstlisting}
MyScene {
    public void trickyParts(int i, int j) {
	/* Make sure that builder syntax is untouched in string
	 * literals */
	String thisString = "Builder(key=vale)";

	/* Make sure that builder syntax doesn't effect the equality
	 * operator. */
	Builder(key==value);

	/* Compare to the actual output outside quotes. */
	Builder(key=value);

	/* Make sure that shape accessors don't mangle string
	 * literal. */
	String x = "#testString";
    }
	
    shapes {
	shape Cylinder(name="cyl1", radius=10cm, 
		       location=(20mi, 30yd), height=50cm);

	// Nested blocks
	if (4 newtons + 2.7e10 forcelb == 6 dynes) {
	    print("Something happened");
	}
    }

    /**
     * Creates some super-special forces.
     */
    forces {
	force SpringForce(actsOn=#cyl1, length=10mi, strength=4);
    }

    properties { }

    onFrame {
	#myShape.set_radius(20);
    }
}
\end{lstlisting}

\subsection*{tests/custom.f}
\begin{lstlisting}
CustomScene {
    properties none;
    forces none;

    shapes {
	shape CustomObject(name="knot", file="scenes/ReefKnot.obj");
    }
}
\end{lstlisting}

\subsection*{tests/rodbendingforce.f}
\begin{lstlisting}
MyScene {
    shapes {
	//shape CustomObject(name="obj1", file="scenes/BunnyCrusher.obj");
	shape SpringChain(name="obj1", vert=(-20,0,0), vert=(-20,10,0), vert=(-30,10,0));
	// Nested blocks
	if (4 newton + 2.7e10 forcelb == 6 dyne) {
	    print("Something happened");
	}
    }

    /**
     * Creates some super-special forces.
     */
    forces {
	force RodBendingForce(actsOn=#obj1, strength=1, angle=0);
	//force GravityForce(gy=-9.8);
    }

    properties { 
	property SemiImplicitEuler(stepSize=0.01);
   }
}
\end{lstlisting}

\subsection*{tests/collision/ee2.f}
\begin{lstlisting}
MyScene {
   shapes {
      shape TriangleShape(name="tri1", vert=(10,0,0), vert=(10,10,0), vert=(10,10,10));
      shape TriangleShape(name="tri2", vert=(20,5,-5), vert=(30,5,5), vert=(20,5,5), velocity=(-1, 0, 0), velocity=(-1, 0, 0), velocity=(-1, 0, 0));
    }

    
    forces {
      force TriangleForce(actsOn=#tri1, youngs=10.0, poisson=.1);
      force TriangleForce(actsOn=#tri2, youngs=10.0, poisson=.1);
    }

    properties { 
        property SemiImplicitEuler(stepSize=0.01);
        property BoundingVolumeHierarchy(margin=0.1);
        property ProximityDetector(proximity=0.1);
        property SpringPenaltyResponder(detector=0, stiffness=1000, proximity=0.1);
   }
}
\end{lstlisting}

\subsection*{tests/collision/ee1.f}
\begin{lstlisting}
MyScene {
   shapes {
      shape TriangleShape(name="tri1", vert=(10,0,0), vert=(10,10,0), vert=(10,10,10));
      shape TriangleShape(name="tri2", vert=(20,5,-5), vert=(30,5,5), vert=(20,5,5), velocity=(-1, 0, 0), velocity=(-1, 0, 0), velocity=(-1, 0, 0));
    }

    
    forces {
      force TriangleForce(actsOn=#tri1, youngs=10.0, poisson=.1);
      force TriangleForce(actsOn=#tri2, youngs=10.0, poisson=.1);
    }

    properties { 
        property SemiImplicitEuler(stepSize=0.01);
        property BoundingVolumeHierarchy(margin=0.1);
        property ProximityDetector(proximity=0.1);
   }
}
\end{lstlisting}


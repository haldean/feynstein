\subsection*{translator/blocks.py}
\begin{lstlisting}
#!/usr/bin/env python3

class Block:
    '''
    Represents a block of code with an identifier. 

    This identifier usually corresponds to either a method or class
    declaration. Tags are used to keep track of the Feynstein context
    which this block exists within (i.e., is it in the shape block,
    the onFrame block, etc.).
    '''

    def __init__(self, block_id, children, tag=None):
        self.block_id = block_id
        self.children = children
        self.tag = tag
        self.name = None

    def block_to_string(self, tab_level=0):
        '''
        Convert a block to Java syntax.
        '''

        rest = '  ' * tab_level + '%s { \n' % (self.block_id)
        for child in self.children:
            if isinstance(child, Block):
                rest += child.block_to_string(tab_level+1)
            else:
                rest += '  ' * (tab_level + 1) + '%s;\n' % child
        rest += '  ' * tab_level + '}\n'
        return rest

    def get_by_tag(self, tag):
        '''
        Returns the first child with the given tag. If it has no
        children with the correct tag, returns None.
        '''

        for child in self.children:
            if isinstance(child, Block) and child.tag == tag: 
                return child
        return None

    def __str__(self):
        return self.block_to_string()
\end{lstlisting}

\subsection*{translator/imports.py}
\begin{lstlisting}
#!/usr/bin/env python3

imports = [
    'feynstein.shapes.*', 'feynstein.forces.*', 'feynstein.*',
    'feynstein.renderer.*', 'feynstein.properties.*', 
    'feynstein.properties.integrators.*', 'java.awt.*', 'java.awt.event.*',
    'javax.media.opengl.awt.*', 'com.jogamp.opengl.util.*', 
    'feynstein.properties.collision.*',
    ]

def get_imports():
    return imports
\end{lstlisting}

\subsection*{translator/parse.py}
\begin{lstlisting}
#!/usr/bin/env python3

import re

from translator import SyntaxException

import blocks, imports, matchers

def split(source):
    '''
    Converts a source file to a series of statements, removing
    comments. blocks.Block identifiers have not yet been removed from the
    expressions at the end of this pass.
    '''
      
    sourceLines = source
    sourceLines = re.sub(matchers.braces, r'\1;',sourceLines)

    # Replace all single line comments before we mangle whitespace
    source = re.sub(matchers.line_comments, '', source)

    # Replace all whitespace by a single space
    source = re.sub(matchers.whitespace, r' ', source)

    # Replace all multiline comments (this must be done after newlines
    # are removed).
    source = re.sub(matchers.multiline_comments, '', source)

    # Place a semicolor after all braces, so that open and close block
    # statements will be split in the following split-on-semicolon.
    source = re.sub(matchers.braces, r'\1;', source)

    # Remove trailing whitespace.
    exprs = [x.strip() for x in source.split(';')]

    exprs2 = exprs
    exprslines = []
    
    lines = sourceLines.split('\n')
    

    #Maps expressions to line numbers and associate it with a java line number.
    javaline = 12
    for index, l in enumerate(lines):
        if re.search(';',l):
            javaline += 1
            exprslines.append([index+1,javaline])
            if re.search('none',l):
                javaline += 1
                exprslines.append([index+1,javaline])

    return exprs, exprslines

def is_open_block(expr):
    '''
    Returns true if an expression opens a block -- i.e., if an
    expression ends with an open-brace.
    '''
    return expr.endswith('{')

def is_close_block(expr):
    '''
    Returns true if an expression closes a block -- i.e., if an
    expression is composed solely of a close-brace.
    '''
    return expr == '}'

def block_id(expr):
    '''
    Given an open block expression, returns the corresponding
    identifier of the block that is opened.
    '''

    if not is_open_block(expr):
        raise SyntaxException('Not a block open statement: %s' % expr)
    return expr.strip('{ ')

def parse(exprs):
    '''
    Convert a series of expressions to a syntax tree. This also
    creates a "root" block which encompasses all parent expressions,
    to ensure that the syntax tree is rooted at a single node.
    '''

    root = blocks.Block(None, make_blocks(exprs), 'root')
    for imp in imports.get_imports():
        root.children.insert(0, 'import %s;' % imp)
    return root

def make_blocks(exprs):
    '''
    Convert a list of expressions into a syntax tree. Called
    recursively to generate a full tree.
    '''

    block_list = []
    i = 0

    while i < len(exprs):
        expr = exprs[i]
        if is_open_block(expr):
            try:
                inner_blocks = 1
                j = i

                # Find the corresponding close-block
                while inner_blocks > 0:
                    j += 1
                    if is_close_block(exprs[j]):
                        inner_blocks -= 1
                    elif is_open_block(exprs[j]):
                        inner_blocks += 1

                # Create a new block, recursively call make_blocks
                # on the statements between the open and close block,
                # and set the new block's children to the result.
                block_list.append(blocks.Block(block_id(expr), make_blocks(exprs[i+1:j])))

                # Move the stack pointer forward to represent all of
                # the statements that were just added to the new
                # block's children.
                i = j
            except IndexError:
                print('Exception encountered while parsing %s' % expr)
                raise SyntaxException('Not enough close block statements. ' +
                                      'You\'re probably missing a close-brace ' +
                                      'or a semicolon.')

        # If it's not an open block statement, just add it to the
        # sequence of expressions we've found thus far.
        elif expr: block_list.append(expr)
        i += 1

    return block_list
\end{lstlisting}

\subsection*{translator/main.txt}
\begin{lstlisting}
public static void main(String[] args) {
    SceneClass scene = new SceneClass();
    canvas.addGLEventListener(new Renderer(scene));

    frame.add(canvas);
    frame.setSize(640, 480);
    frame.setUndecorated(true);
    frame.setExtendedState(Frame.MAXIMIZED_BOTH);
    frame.addWindowListener(new WindowAdapter() {
	    public void windowClosing(WindowEvent e) {
		System.exit(0);
	    }
	});
    frame.setVisible(true);
    animator.start();
    canvas.requestFocus();
}\end{lstlisting}

\subsection*{translator/matchers.py}
\begin{lstlisting}
#!/usr/bin/env python3

import re

def method_for(method):
    '''
    Return a private instance method for a given method signature.
    '''

    return '@Override public void %s' % method

def for_unit(unit_name):
    '''
    Returns a pattern that matches a unit value, with the value in
    group 0.
    '''

    pattern = r'(\d+\.?\d*(?:[eE]\d+)?)\s*%s(?=[^a-zA-Z])' % unit_name
    return re.compile(pattern)

# Matches strings of whitespace
whitespace = re.compile(r'[\n\t ]+')

# Matches open or close braces and puts the matched brace in group 0
braces = re.compile(r'([{}])')

# Matches single-line comments of the // variety
line_comments = re.compile(r'//.*')

# Matches multiline comments, but only after all newlines have been
# removed.
multiline_comments = re.compile(r'/\*.*?\*/')

# Matches lowercase keywords.
keyword = re.compile(r'[a-z]+')

# Matches identifiers.
identifier = re.compile(r'[a-zA-Z][a-zA-Z0-9_]*')

# While this is not definitive proof of the usage of builder syntax,
# it is a good guess. A negative match means that an expression is
# conclusively not in builder syntax. A positive match means that an
# expression probably uses builder syntax.
builder_hint = re.compile(r'[a-zA-Z][a-zA-Z0-9_]*\(' +
                          r'[a-zA-Z][a-zA-Z0-9_]*=[^=]')

# Matches an identifier followed by a paren-wrapped string. First
# group is the identifier, second group is the argument list.
outer_parens = re.compile(r'([a-zA-Z][a-zA-Z0-9_]*)\((.*)\)')

# Matches a shape accessor.
accessor = re.compile(r'\#([a-zA-Z][a-zA-Z0-9_]*)')

# Definitions for block method translation.
block_translations = {
    'shapes': method_for('createShapes()'),
    'forces': method_for('createForces()'),
    'properties': method_for('setProperties()'),
    'onFrame': method_for('onFrame()'),
    }
\end{lstlisting}

\subsection*{translator/compile.py}
\begin{lstlisting}
#!/usr/bin/env python3

import glob,re, os, subprocess, sys, translator

def get_classpath():
    libs = ':'.join(glob.glob('./libs/*.jar'))
    return '.:%s:%s' % (os.getcwd(), libs)

def get_jvm_vars():
    return '-Djava.library.path=/lib/:libs/'

def error_gen(err,javamap):
    err = str(err)
    err = err.split('\n')
    errorlist = []
    errormessage = 'Feynstein Errors:\n'
    for e in err:
        if re.search('java:',e):
            m = re.search('[0-9]+',e)
            errorlist.append(int(m.group()))
    
    if len(errorlist) > 0:
        for er in errorlist:
            for j in javamap:
                
                if j[1] == er:
                    errormessage=errormessage+'Syntax error in line ' + str(j[0]) + '.\n'

    return errormessage

def compile_feynstein(infile):
    java_source,javamap = translator.main(infile)
    p = subprocess.Popen(['javac', '-classpath', get_classpath(), 
                     '-Xlint:unchecked', java_source],stdout = subprocess.PIPE)
    out,err = p.communicate()
    
    
    
    print(error_gen(out,javamap))

    return java_source

def run_feynstein(infile):
    package = '.'.join(infile.replace('.', '/').split('/')[:-1])
    subprocess.call(['java', '-classpath', get_classpath(), 
                     get_jvm_vars(), package])

if __name__ == '__main__':
    infile = sys.argv[1]
    if len(sys.argv) > 2 and sys.argv[2] == 'compile':
        compile_feynstein(infile)
    else:
        run_feynstein(compile_feynstein(infile))
\end{lstlisting}

\subsection*{translator/syntax.py}
\begin{lstlisting}
#!/usr/bin/env python3

from translator import SyntaxException
import matchers

def check_syntax(root):
    '''
    Ensure that no syntactic rules are broken.
    '''

    scene = root.get_by_tag('scene')
    check_blocks(scene)

def check_blocks(scene):
    '''
    Ensure that all of the correct blocks are present.
    '''

    if not scene.get_by_tag('shapes'):
        raise SyntaxException('No shapes block found.')
    if not scene.get_by_tag('forces'):
        raise SyntaxException('No forces block found.')
    if not scene.get_by_tag('properties'):
        raise SyntaxException('No properties block found.')
    if not scene.get_by_tag('onFrame'):
        print('Warning: no onFrame block found. ' +
              'An empty frame update method will be used.')
        scene.children.append('%s() {}' % matchers.method_for('onFrame'))
\end{lstlisting}

\subsection*{translator/units.py}
\begin{lstlisting}
#!/usr/bin/env python3

import blocks, matchers, re

conversions = {
    # Length
    'km': 1000,
    'm': 1,
    'cm': 0.01,
    'mm': 0.001,

    'mi': 1609.34,
    'yd': 0.9144,
    'ft': 0.3048,

    # Time
    'year': 3.154e7,
    'month': 2.628e6,
    'week': 604800,
    'day': 86400,
    'hour': 3600,
    'min': 60,
    'sec': 1,
    'ms': 0.001,
    'microsec': 1e-6,

    # Mass
    'tonne': 1000,
    'kg': 1,
    'g': 0.001,

    'ton': 1016.0469,
    'lb': 0.4536,
    'oz': 0.028349,

    # Force
    'forcelb': 4.448,
    'newton': 1,
    'dyne': 1e-5,
    }

unit_matchers = {}
for unit in conversions.keys():
    unit_matchers[unit] = matchers.for_unit(unit)

def translate_units_for_expr(expr):
    if expr and re.search('\d', expr):
        for unit, m in unit_matchers.items():
            expr = re.sub(m, r'(\1*%f)' % conversions[unit], expr)
    return expr

def translate_units(block):
    for i, expr in enumerate(block.children):
        if isinstance(expr, blocks.Block):
            expr.block_id = translate_units_for_expr(expr.block_id)
            translate_units(expr)
        else:
            block.children[i] = translate_units_for_expr(expr)
\end{lstlisting}

\subsection*{translator/translator.py}
\begin{lstlisting}
#!/usr/bin/env python3

'''
The Feynstein compiler. (Mostly) written early one morning, with the
help of lots of beer. I'm going to bed now, I think.
'''

class SyntaxException(Exception): 
    '''
    An exception raised when a syntax error is detected when parsing a
    Feynstein source file.
    '''
    pass

import blocks, matchers, os, parse, re, sys, syntax, translate

def create_java(root):
    '''
    Create a Java source string from a translated root block.
    '''

    return '\n'.join([str(x) for x in root.children])

def src2blocks(source):
    exprs,lines = parse.split(source)
    root = parse.parse(exprs)
    translate.translate(root)
    syntax.check_syntax(root)
    return root,lines

def feync(source, path):
    '''
    Compile a source file, returning a tuple with the Java source code
    and the name of the scene.
    '''

    root,lines = src2blocks(source)

    package = os.path.dirname(path).replace('/', '.')
    root.children.insert(0, 'package %s;' % package)

    scene_name = root.get_by_tag('scene').name

    main_file = "%s/main.txt" % os.path.dirname(__file__)
    with open(main_file) as f:
        main_method = f.read().replace('SceneClass', scene_name);
        if main_method:
            root.get_by_tag('scene').children.append(main_method)

    return create_java(root), scene_name, lines

def main(infile):
    '''
    Compile infile and return the filename of the generated Java
    source code.
    '''

    with open(infile) as f:
        source = f.read()

    source, scene_name,lines = feync(source, infile)
    output_file = '%s/%s.java' % (os.path.dirname(infile), scene_name)

    with open(output_file, 'w') as f:
        f.write(source)
    print('Compiled to %s' % output_file)
    
    return output_file,lines

if __name__ == '__main__':
    infile = sys.argv[1]
    main(infile)
\end{lstlisting}

\subsection*{translator/translate.py}
\begin{lstlisting}
#!/usr/bin/env python3

from translator import SyntaxException

import blocks, matchers, re, units

def translate_block_id(block):
    '''
    Replaces block identifiers with their corresponding replacement,
    as defined in matchers.block_translations. Any block identifier
    not in matchers.block_translations remains untouched.
    '''

    if block.block_id in matchers.block_translations:
        # If we're translating this block_id, set its tag to the
        # block_id specified in the Feynstein source.
        block.tag = block.block_id
        block.block_id = matchers.block_translations[block.block_id]

def translate_root_id(block):
    '''
    Replaces the block identifier for a scene block with the
    corresponding class definition.
    '''

    block.name = block.block_id
    block.block_id = 'public class %s extends Scene' % block.block_id
    block.tag = 'scene'

def translate_ids(block, is_root=True):
    '''
    Translate all block identifiers recursively.
    '''

    # Keep track of whether we've seen a block; the root must contain
    # exactly one block (the scene).
    block_seen = False

    for expr in block.children:
        if isinstance(expr, blocks.Block):
            if is_root and block_seen:
                raise SyntaxException('Source may only contain one top-level block. ' +
                                      'You cannot specify more than one scene in one file.')
            block_seen = True
            if is_root:
                translate_root_id(expr)
            else:
                translate_block_id(expr)

            # Recursively call on this block.
            translate_ids(expr, is_root=False)

    if is_root and not block_seen:
        raise SyntaxException('Source must contain a scene.')

def disperse_tags(root):
    '''
    Trickle down tags so that all blocks within other blocks in a
    scene are tagged with their parents' tags. This is necessary for
    later translation steps; for example, the "shape" directive has a
    different meaning if it is in the "shapes" block.
    '''

    def set_child_tags(block, tag):
        '''
        Set the tag for block and all of its children.
        '''

        block.tag = tag
        for expr in block.children:
            if isinstance(expr, blocks.Block):
                set_child_tags(expr, tag)

    # If this isn't a scene block, assume that it's the root and find
    # the scene block in its children.
    if root.tag != 'scene':
        root = root.get_by_tag('scene')
        if not root:
            raise SyntaxException('Source must contain a scene.')
    
    for child in root.children:
        if isinstance(child, blocks.Block):
            set_child_tags(child, child.tag)

def is_builder(expr):
    '''
    Return a best-guess for whether or not expr represents an
    expression that contains builder syntax. Note that the result of
    this method is not guaranteed to be correct.
    '''
    match = re.search(matchers.builder_hint, expr)
    if match and expr.count('"', 0, match.start()) % 2 == 0:
        return True
    return False

def translate_builder(expr, parent_ref=False):
    '''
    Given an expresion in builder syntax, output the corresponding
    Java translation.
    '''

    # Get the class name and parameter list
    match = re.search(matchers.outer_parens, expr)
    class_name, param_list = match.groups()
    
    params = []
    nest_level = last_boundary = 0
    in_quote = False

    # Process input one character at a time; this is essentially a PDA
    # to determine the boundaries between arguments at the parent
    # level, and ignore commas that correspond to other method calls.
    for i in range(len(param_list)):
        char = param_list[i]

        if in_quote and char == '"':
            in_quote = False
        elif char == '"':
            in_quote = True
        elif char == '(':
            nest_level += 1
        elif char == ')':
            nest_level -= 1
        elif (char == ',' or char == ')') and nest_level == 0 and in_quote == False:
            params.append(param_list[last_boundary:i])
            last_boundary = i

    # Add the remainder, removing any trailing commas.
    params.append(param_list[last_boundary:].strip(','))

    # Process each key-value pair
    for i, p in enumerate(params):
        attr, eql, value = p.strip(' ,').partition('=')

        # If the value starts with a parenthesis, we remove a set of
        # parentheses from the outside of this statement. This is for
        # tuple invocation of multiparametered methods.
        if value.startswith('('):
            value = value[1:-1]

        params[i] = 'set_%s(%s)' % (attr, value)

    # If a parent reference is required, add one. This is usually only
    # necessary for properties.
    if parent_ref:
        constructor_args = 'this';
    else:
        constructor_args = '';

    # Return a formatted string, leaving the prefix to the builder
    # syntax intact.
    return '%s(new %s(%s)).%s.compile()' % (
        expr[:match.start()], class_name, constructor_args, '.'.join(params))

def translate_builders(block):
    '''
    Recursively translate builder syntax.
    '''

    for i, expr in enumerate(block.children):
        if isinstance(expr, blocks.Block):
            translate_builders(expr)
        else:
            if is_builder(expr):
                block.children[i] = translate_builder(expr, block.tag == 'properties')

def translate_block_directives(block):
    '''
    Recursively translate block directives, like "shape" or "force"
    in the "shapes" or "forces" block, respectively.
    '''

    for i, expr in enumerate(block.children):
        if not isinstance(expr, blocks.Block):
            # If an expression starts with "shape" in the shapes
            # block, it is an add shape directive, and likewise for
            # "force" in the forces block.
            if block.tag == 'shapes' and expr.startswith('shape '):
                block.children[i] = 'addShape(%s)' % expr[len('shape '):]
            elif block.tag == 'forces' and expr.startswith('force '):
                block.children[i] = 'addForce(%s)' % expr[len('force '):]
            elif block.tag == 'properties' and expr.startswith('property '):
                block.children[i] = 'addProperty(%s)' % expr[len('property '):]
        else: translate_block_directives(expr)

def translate_shape_accessors(block):
    '''
    Recursively translate shape accessors.
    '''

    for i, expr in enumerate(block.children):
        if not isinstance(expr, blocks.Block):
            # Split the string by double quotes, then only replace
            # accessors in the even bits. The odd bits correspond to
            # string literals in this expression, and should not be
            # touched.
            bits = expr.split('"')
            for j, b in enumerate(bits):
                if j % 2 == 0:
                    bits[j] = re.sub(matchers.accessor, r'getShape("\1")', b)
            block.children[i] = '"'.join(bits)
        else: translate_shape_accessors(expr)

def translate_empty_blocks(block):
    '''
    Recursively translate the "block none;" construct.
    '''

    for i, expr in enumerate(block.children):
        if not isinstance(expr, blocks.Block):
            if expr.endswith(' none'):
                block.children[i] = blocks.Block(expr[:-5], [])
        else:
            translate_empty_blocks(expr)

def translate(root):
    '''
    Perform all translation tasks on a syntax tree.
    '''

    translate_empty_blocks(root)
    translate_ids(root)
    disperse_tags(root)
    translate_shape_accessors(root)
    translate_builders(root)
    translate_block_directives(root)
    units.translate_units(root)
\end{lstlisting}


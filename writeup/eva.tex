Because there are so many components to Feynstein above and beyond
merely translating the code, much of the testing methodology was
geared towards validating the implementation of the physics of the
program. Testing for these elements involves rendering actual scenes
written in Feynstein in order to visually verify that the math in the
code is working. The Feynstein tests for built-in forces were modelled
after XML files used by Sam Ainsley in her physical simulation
project. As new forces and physics functions are added to the program,
the suite of visually-verifiable test programs can expand accordingly,
and is re-run in order to assure that the new code has not caused an
unintended change in the functioning of the old.

The Feynstein translator test consists of a Feynstein source code and
corresponding output in java. By running the code through a new
version of the translator and saving it before it is compiled and run,
one can verify that the new translator performs equivalent
transformation. Once this new output is compiled and executed to
ensure its validity, it can be saved as the test for the next
generation Feynstein translator.

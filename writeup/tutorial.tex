\subsection{Simple Program}

Let's start with a basic Feynstein program.

\begin{lstlisting}{caption=ShapeScene.f}
ShapeScene {
   shapes {
     shape Sphere(name="sphere1", radius=20cm, 
       location=(0,0,0));
   }
 }
\end{lstlisting}

This is as basic a program you can have in Feynstein; a single object
with no forces acting on it. This will render a scene to a window on
your screen; this window will allow you to pan and tilt using your
mouse and keyboard, just like most popular 3D editing software. There
are a few things to note, just from this short code example:

\begin{enumerate}
\item The source file has a \code{.f} extension. This is necessary
  for the compiler and interpreter to compile and run your Feynstein
  source file.
\item The scene has a name: \code{ShapeScene}. This name needs
  to match the name of the source file, and it is the name you will
  pass in to the interpreter to render your scene.
\item There is block called \code{shapes}. Within this block, the user can
  define the shapes she would like to place in the scene. Anything
  goes in this block, though; anything not preceded by the \code{shape}
  keyword will be treated as standard Java code.
\item There is a \code{shape} keyword. The \code{shape} keyword
  defines a type in Feynstein -- any kind of shape you add to the
  scene inherits from \code{shape}. In the \code{shapes} block, any
  \code{shape} defined on a line by itself is automatically added to
  the scene.
\end{enumerate}

There is also a syntactical oddity here, as well. Most Java users will
be right at home using Feynstein, as almost all of the syntax is
borrowed from Java. However, one large irritant of defining a 3D scene
in Java is just how many parameters you need to specify for every
object. Each object needs an identity, size and location at the
minimum, and to make matters worse, both the size and location of
every object have dimensions. Feynstein adds a new way to create
objects, called “Builder Syntax”, which lets you use key-value pairs
to initialize an object. You can see this in \code{ShapeScene.f}; we
create a new sphere by calling \code{shape Sphere(name=”sphere1”,
  radius=20cm, location=(0,0));}. Builder syntax allows you to set
only the properties you want, allowing the rest to remain at their
default value. In addition, Feynstein recognizes most standard units
for length, mass, force, velocity and acceleration; commonly used
S.I. (meter, centimeter, gram, Newton, etc) and Imperial units (foot,
inch, mile, pound) are all supported. For a full list of supported
units, consult the Feynstein User Manual.

There are many shapes a user can add that come built into Feynstein,
and it's also really easy to import a 3D model from somewhere
else. The included shapes are shown in the table below. Note that
every shape has the parameters \code{location} and \code{mass}.

\begin{table}\centering
  \begin{tabular}{r|l}
    \textbf{Shape} & \textbf{Parameters} \\ \hline
    RectangularPrism & height, width, location, mass, name \\
    Cylinder & height, radius1, radius2 (optional), location, mass, name \\
    Sphere & radius, location, mass, name \\
    Tetrahedron & edges, location, mass, name \\
    Plane & normal, location, mass, name \\
    CustomObj & file, location, mass, name \\
  \end{tabular} 
\caption{Built-in shapes and their supported parameters}
\label{tab:shapestut}
\end{table}

You can also import objects created in other 3D programs into
Feynstein if they're in a .OBJ format. Adding OBJ files is easy; you
just add a \code{CustomObj} shape in the same way you'd add any shape, and
you pass the file name of model as the \code{file} parameter. Feynstein
imports it for you, and you can set the \code{location} and \code{mass}
properties of the shape just like you would for any object.

\subsection{Forces}
No simulation package would be complete without a full array of forces
at its users' disposal, and Feynstein is no exception to this
rule. With eight built-in forces, the user can specify almost any
real-world scenario by just specifying parameters. An example program
that includes forces is shown below.

\begin{lstlisting}{caption=ForceScene.f}
ForceScene {
   shapes {
     shape Sphere(name="sphere1", radius=20cm, 
       location=(0,0,0));
   }

    forces {
      force GravityForce(gx=0, gy=0, gz=-9.81);
      force SpringForce(restLength=1m, k=1, 
        actsOn=#sphere1, fixedAt=(0, 0, 2m));
    }
 }
\end{lstlisting}

This scene defines two forces using the same builder syntax that we
used to define shapes earlier. We place the same sphere that we had
earlier in our scene, but then we attach it to a spring. This
simulation would result in a video of a sphere bouncing up and down
before it eventually comes to rest about one meter above where it
began. Each type of force has a different set of parameters; in this
scene we've used GravityForce (which allows you to configure the
direction and strength with which the gravitational force acts) and
SpringForce (which allows you to configure the rest length, spring
force, the location of its fixed end, if any, and the objects upon
which it acts).

These two forces are the most intuitive, but there are many forces you
might be interested in using. In addition to GravityForce and
SpringForce, Feynstein has the following forces at your disposal. Note
that for all forces, the parameter \code{actsOn} specifies an object (or
objects) which that force acts upon.

\begin{description}
\item[DampingForce] A DampingForce is a frictional force which resists
  motion, and is a function of mass. You can attach a DampingForce to
  any object, and that object's motion will be resisted. The magnitude
  of its resistance is given by a single parameter, \code{lambda}.

\item[RodBendingForce] A RodBendingForce acts upon three particles, which
  together act like a hinge, to resist the bending of these
  particles. It has three parameters: the rest length between the
  first pair of particles \code{restLength1}, the rest length between the
  second pair of particles \code{restLength2}, and the rest angle \code{theta}.

\item[TriangleForce] This is a force that acts against the
  deformation of a triangle. Given the side lengths of the triangle,
  the stiffness of the resistance \code{stiffness}, and the
  compression-to-expansion ratio \code{poisson}, it will act against any
  force that attempts to skew the triangle in any way.

\item[SurfaceBendingForce] A SurfaceBendingForce is a constraint force that
  resists the bending of a four-particle surface along its diagonal.
  These four particles are arranged in two triangles which share an
  edge. The strength of the force is a function of the angle between
  these two triangles, and is parameterized by the resistance to
  bending, \code{stiffness}, rest angle of the two triangles, \code{theta}, and
  the shape of both triangles.

\item[ContactForce] A ContactForce is derived from a SpringForce and acts
  upon two triangles to resist a collision between them. An equal and
  opposite force is applied to both triangles which pushes them apart,
  which is analogous to a collision. A contact force is configured by
  a spring force stiffness \code{stiffness} and a minimum distance
  (\code{minDist}) that must be maintained between the colliding pairs.
\end{description}

If none of these forces satisfy your simulation needs, it's easy to
define your own, as well! We'll discuss how to define your own forces
in the Advanced Topics subsection.

\subsection{Frame Updates}

While you're rendering your scene, you get qualitative output in the
preview window. If you're modelling a scientific experiment, though,
you'll often need much more precision than that; what if you want to
know exactly where an object is in a given frame? That's where frame
update methods come in.

The frame update method of a scene is a block that is executed every
time the scene is stepped forward in time, but before it is
rendered. This gives the programmer an opportunity to observe or
change some of the properties of the objects in the scene before the
next frame is generated. Let's take the ball-on-a-spring example again
to see how this might be useful.

\begin{lstlisting}{caption=ForceScene.f}
import java.io.*;

ForceScene {
    FileWriter writer;

    static {
      try {
        writer = new FileWriter(“output.csv”);
      } catch (IOException) {
        // handle error
      }
    }

    shapes {
      shape Sphere(name="sphere1", radius=20cm, location=(0,0,0));
    }

    forces {
      force GravityForce(gx=0, gy=0, gz=-9.81);
      force SpringForce(restLength=1m, k=1, actsOn=#sphere1, fixedAt=(0, 0, 2m));
    }

    onFrame {
      writer.write(String.format(“\%d,\%d\n”, time, #sphere1.getZ()));
    }
 }
\end{lstlisting}

This code generates the same visualization we had before -- a sphere
bobbing on a spring -- but this time it does something extra; for
every frame that is rendered, it also writes a record to a CSV file
with the time that has elapsed and the height of the sphere. If we
then wanted to go away and do further processing on that data in
Matlab or Mathematica, it's in easy, machine-readable format.

There's lots of new syntax and features in this snippet, so we'll
start at the top. The first thing you'll notice is the \code{import}
statement -- imports work exactly the same in Feynstein as they do in
Java, and you have the full Java standard library at your
disposal. You can also see we've defined a scene-wide variable,
outside the scope of any of our blocks. Anything that isn't in a
predefined block is treated as if it were in a standard Java class. In
this case, that means our \code{FileWriter} is an instance variable of our
scene.

There's also two new blocks: \code{static} and
\code{onFrame}. \code{static} is called after the system has been
initialized and properties have been set set, but before any shapes or
forces are created. This is a good time to read input from the user,
if any, or to initialize instance variables you want to use in other
blocks. That's exactly what we do here, initializing our
\code{FileWriter} so we can write to it during \code{onFrame}.

\code{onFrame} is the frame update method we discussed in the introduction
to this subsection -- it is the method that gets run after all of the
forces are applied to the shapes in the scene, but before the scene is
rendered. Here, you can set or read any of the properties of a shape;
in our case, we're accessing the height of our sphere in the
scene. Within the onFrame block, you have access to a special
variable, \code{time}, which represents the number of milliseconds that
have passed in scene time.

To get the properties of an object, you have to have a reference to
the object itself, but since we created the object in the shapes
block, we didn't store a reference to it in a variable. Instead, we
access it using its name with the \#-operator. We then can treat it
like a regular object and call instance methods on it, like
\code{getZ()}.

\subsection{Integration Methods and Error Bounds}

Unfortunately for the users of any simulation system, physical
simulation isn't perfect. While, given the right parameters, it can
approximate the real world extraordinarily well, sometimes it fails to
do so quite dramatically. The most common source of failure in
physical simulation is the systematic error (or ``instability'')
inherent in time stepping.

Time stepping is the process by which the simulator figures out the
location and momentum of objects in the next frame, based on the
forces acting on them and their location and momentum in the current
frame. The simulator uses various numerical methods for solving large
systems of equations to project the objects into the next frame, and
each of these comes with its own drawbacks. As a user of Feynstein,
you get to choose which time stepping methods (also called an
“Integration Method”) you would like to use to render your scene. You
can also place limits on the error in the scene, so if the error ever
passes a certain threshold, rendering is halted and the user is shown
an error message.

Feynstein comes with two different time-stepping methods, each with
advantages and disadvantages. These are all properties; for example,
if I wanted to use Velocity Verlet with a step size of 2 milliseconds,
I would add \code{property VelocityVerlet(stepSize=2ms);} to my
\code{properties} block. It is important to note that the stability of
any method is dependent upon the step size used; a larger step size
means lower stability, and vice-versa.

\begin{description}
\item[SemiImplicitEuler] The semi-implicit Euler method of time integration
  is, as the name suggests, a midpoint between explicit and
  implicit. Unlike implicit Euler, it uses the location in the
  previous frame to calculate the velocity in the next frame, giving
  it a source of error (especially when you have fast-moving things in
  your scene). However, it then uses its estimate for the velocity in
  the next frame to calculate its next position, unlike explicit
  Euler, which only uses its knowledge of the current frame. This is a
  good trade-off between speed and stability, and is a good choice for
  everyday rendering tasks.

\item[VelocityVerlet] Although the semi-implicit Euler and velocity
  Verlet techniques are similar, velocity Verlet is more accurate over
  long periods of time. Verlet integration updates velocities in two
  stages, and uses the intermediate velocity to calculate new particle
  positions.
\end{description}

\subsection{Collision Handling}
Collision handling in implemented in Feynstein in through two discrete
modules: detection and response. You can add a collision detector to
your scene without necessarily having a collision responder, but all
collision responders need a detector to find collisions to respond
to. Collision handling doesn't need to be tied to specific
objects -- detectors and responders will check all objects in a scene if
they are defined.

\subsubsection{Collision Detection}
Feynstein supports two predefined options for collision detection:
proximity-based detection with \code{ProximityDetector} and a more
thorough, continuous method, \code{ContinuousTimeDetector}, which uses
time steps to check for collisions. Both of these methods extend the
abstract \code{NarrowPhaseDetector} class because both are part of the
narrow phase of collision detection, where all possible collisions
between objects in a scene are monitored.

Broad-phase collision detection constitutes a helpful, but not
mandatory, optimization for narrow-phase detection: broad-phase
detection uses a \code{BoundingVolumeHierarchy} to store probable
collisions based on a VolumeHierarchy heuristic. To use this
optimization method, add a \code{BoundingVolumeHierarchy} property to
your properties list, before you specify a \code{ProximityDetector}
property. Note that just a \code{BoundingVolumeHierarchy} isn't enough
to detect collisions: you also need to have a narrow-phase detector.

\begin{lstlisting}
    properties {
      property BoundingVolumeHierarchy(margin=0.1, 
             type=BoundingVolumeHierarchy.AABB);
      property ProximityDetector(proximity=0.1);
    }
\end{lstlisting}

\subsubsection{Collision Response}

Feynstein offers two predefined methods of collision response,
extending the abstract \code{CollisionResponder}: the
\code{SpringPenaltyResponder}, which uses spring forces of a high
strength to model elastic collisions between shapes in a scene, and
the \code{ImpulseResponder}, which applies small impulses to
neighboring objects to prevent them from colliding. Both responders
must have a \code{CollisionDetector} to function properly; failing to
define one will create a compile-time error. Detectors are indexed by
the order in which they appear in your properties block, starting with
0, and this integer index is the value for the \code{detector}
parameter in both \code{CollisionResponder}s.

The \code{SpringPenaltyResponder} takes a proximity (how close objects
must be before it responds to a collision) and a strength for the
spring force as parameters, while the \code{ImpulseResponder} takes
only a number of iterations over which to apply impulses to colliding
objects. All of these parameters have default values, so if you're not
sure how precise you want your collision responders to be, specifying
only a detector in your builder syntax will cause Feynstein to operate
with the values we've defined.

\begin{lstlisting}{caption=CollisionScene.f}
CollisionScene {
    shapes {
      shape Sphere(name="sphere1", radius=20cm, location=(0,0,0));
    }

    forces {
      force GravityForce(gx=0, gy=0, gz=-9.81);
      force SpringForce(restLength=1m, k=1, actsOn=#sphere1, fixedAt=(0, 0, 2m));
    }

    properties {
      property SemiImplicitEuler(stepSize=0.01);
      property BoundingVolumeHierarchy(margin=0.1, type=BoundingVolumeHierarchy.AABB);
      property ProximityDetector(proximity=0.1);
      property SpringPenaltyResponder(detector=0, proximity=0.1, stiffness=1000)
 }
\end{lstlisting}

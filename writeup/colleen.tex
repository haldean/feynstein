\subsection{Development Process}
Two aspects of our development process that I believe contributed to
the project's success were the modularity of our language and compiler
and our decision to build our language and our code base in
increments. There were, of course, general aspects of the language
that had to be decided on early in the project, such as the
overarching structure of a Feynstein file, and these were discussed
and voted on at group meetings. Our Language Guru defined a hierarchy
of discrete features for inclusion, so that we could start out with a
small but functional code base and build on it to include as many
possibilities as possible for our languages users. (See section on
Language Evolution for an in-depth description of this process.)
Because we divided our language into sections in accordance with
physical elements--there are shapes (masses), forces, and properties
such as time integration and collision handling--the modules of our
translator could be designed, developed, and tested independently of
each other, which decreased the dependencies in our language and
allowed the maximum flexibility for group members to work at their own
paces to finish the sections of the project assigned to them.

\subsection{Roles and Responsibilities}
While our language was modular and easily separated into discrete sections,
the responsibilities of the project's group members were far less
distinct than we expected and than the course materials' guidlines
suggested they be. I believe this unsystematic distribution of work
speaks to the flexibility of our project group, but certain areas of the
project may have run more smoothly if we had made firm decisions
about what our group roles constituted early in the course of the project.

\begin{description}
\item[Colleen McKenzie: Project Manager] I was responsible for
  organizing meetings and taking minutes at group meetings. I also
  handled updating the group's project wiki and Google Calendar with
  minutes, resources, events and deadlines, and other changing 
  information pertaining to our language and its
  implementation. Monitoring group members' progress was also my
  responsibility, and I regularly checked in with group members to
  keep track of progress in each contributor's area of language
  development and implementation. This monitoring also included the
  setting (and resetting) of deadlines and querying group members
  about their progress. Finally, I was responsible for proofreading
  all documentation (typeset by our Systems Architect) and either 
  correcting errors or communicating issues to the Language Guru.

\item[Samantha Ainsley: Language Guru] Samantha was responsible for
  providing the conceptual organization of the language--she was our
  essentially our Physics Guru, and she used her expertise to lead us
  to consistent and logical decisions about the design of our
  language. She provided reference materials to give us an idea of how
  other languages and libraries implemented functionality similar to
  Feynstein's and was the designated point person for questions
  related to such matters. Further, Sam was largely responsible for
  executive decisions about which features our language would and
  wouldn't support.

\item[Will Brown: Systems Architect] Will was responsible for defining
  the structure of the Feynstein compiler code package, which is
  broken up into modules for translation and for processing different
  simulation aspects. In addition, Will was entirely responsible for
  the translation portion of the compiler: he worked with our Language
  Guru's design to ensure that our language was properly translated
  into Java code built on the compiler architecture he designed. Will
  also managed the repository for the group's code and managed the
  documentation process: he was responsible for compiling and
  typesetting all of our group's written material.

\item[Rob Post: Systems Integrator] Rob was responsible for
  researching information we used to design and add features to our
  rendering module, and make decisions about what features and options
  for visual output we would provide users of our language. Rob also
  collaborated with the Systems Architect to implement robust
  compile-time error handling in our translator.

\item[Eva Snow: Tester and Validator] Eva was responsible for
  implementing unit tests designed to make sure our language properly
  supported all the features defined in our manual and ensuring that
  the physical features of our language had been implemented
  correctly.
\end{description}

\subsection{Implementation Stylesheet}
Our group did not have an implementation stylesheet. Our development
process included group members reading through project code before
contributing more, and because our Language Guru and Systems Architect
defined a robust API while laying the groundwork for the compiler,
members contributing new modules were able to sufficiently match the
project's existing code.

\subsection{Project Timeline}
\begin{description}
\item[1/24] We decided on and assigned roles for each team member,
  made logistical decisions (e.g. weekly meetings), and decided on
  physical simulation as the domain for our language.

\item[1/30] Group members gained familiarity with the language domain,
  including higher-level physical concepts we would need to implement.

\item[2/6] We developed and recorded milestones for our project and
  started working on a general language specification to guide future
  language developments and additions.

\item[2/13] With a few tweaks, we made our earlier language
  specification final. By this point our builder syntax is
  well-defined.

\item[2/20] Our language whitepaper is finished at this point.

\item[2/27] We defined more specific milestones for implementing the
  compiler, and we identified and accounted for dependencies in the
  structure of the language and the compiler.

\item[3/29] By this point, the language reference manual and tutorial
  were complete. We had a solid specification and a grammar for our
  language, both of which were very close to the grammar of our
  finalized language.

\item[4/3] Translation for Feynstein-specific syntax was implemented,
  and early translator testing was implemented alongside the
  translator. Our git repository was up and running, and all members
  were able to commit code. We implemented basic functionality of our
  renderer so that it could could display a basic shape to the
  screen. Mouse and keyboard commands were implemented with the
  rendered so that users could change the view of their scene.

\item[4/10] Automated compilation and running of Feynstein files was
  implemented by this point. The modules in our compiler were divided
  up discretely among group members.

\item[4/18] Time-stepping was implemented, but not integrated into
  Feynstein yet.

\item[4/24] We implemented basic discretization of geometry: at this
  point we had fully functional objects in our language, and these
  could be operated on by force constructs. Basic forces (gravity,
  springs) were also implemented.

\item[5/3] All forces were implemented by this point.

\item[5/5] Basic force testing was completed and collision detection
  was implemented by this point.

\item[5/6] Collision response was implemented, and the translator was
  extended to handle errors in more specific and descriptive ways.
\end{description}
\subsection{Project Log}
Please see Appendix \ref{sec:projectlog} for the (rather lengthy)
project log.
